% This starts the document; the "npsreport" style is really a modified
% report style. Feel free to use the options explained in the technical
% report NPS-CS-11-011, included under the doc/ directory.
\documentclass[twoside,thesis]{npsreport}

%
% Put extra packages you may need to customize your thesis
%
\usepackage{doc,lipsum} % provides \BibTex and \lipsum macros, for demos

%
% For Example: you might find one of these useful:

%\usepackage{epstopdf}        % to use .eps files for figures
%\usepackage{bm}              % bold math if you need bold greek letters
%\usepackage{glossaries}      % see http://en.wikibooks.org/wiki/LaTeX/Glossary
%\usepackage{asymptote}       % for graphics
% The asymptote package allows for very nice graphics and figures
% Proper usage requires additional information located at:
% http://asymptote.sourceforge.net/
% See the gallery at this URL for examples

%\usepackage{placeins}        % float placement
% Provides \FloatBarrier which keeps figures/tables in the same section.
% LaTeX sometimes moves them to fill up pages.
% http://ftp.math.purdue.edu/mirrors/ctan.org/macros/latex/contrib/placeins/placeins-doc.pdf

%\usepackage[numbered]{mcode} % matlab code
% The mcode package must be separately downloaded.
% http://www.mathworks.com/matlabcentral/fileexchange/8015-m-code-latex-package

%\usepackage{flafter}         % float placement
% Ensures that figures/tables do not appear in the document before
% they are referenced in the text.

% This package lets us build references that appear after the Executive Summary
\usepackage{bibunits}


\title{[Title]}

% Student info
\author{[Author Name]}
\rank{[Rank, Service]}    %\rank{Civilian} % if you don't have a rank
\degree{Master of Science in [Degree]}
\degreeabbreviation{MS}   % Should be MS, MBA or MA
\prevdegrees{[B.S., My Old School, Year]} % previous degree

% Department info
\department{Department of [Department]}
\thesisadvisor{[Primary Advisor]}
\secondreader{[Second Reader]}
\departmentchair{[Department Chair]}

% The date you are graduating:
\degreedate{[Month Year]}

% See Thesis processor's release form for approved distribution statements.
\distribution{Approved for public release; distribution is unlimited}

% Your abstract.  New paragraphs start after an empty line.
\abstract{%
\lipsum[1] % example text, remove me
}

% Switch the below lines around, if FOUO
\securitybanner{}  %\securitybanner{FOR OFFICIAL USE ONLY}

%
% Mandatory fields for the SF298.
%
\ReportType{Master's Thesis}
\ReportDate{MM-DD-YYYY}         % for a thesis, graduation date
\SponsoringAgency{N/A}          % really, for technical reports
\DatesCovered{MM-DD-YYYY to MM-DD-YYYY}
\ReportClassification{Unclassified}
\AbstractClassification{Unclassified}
\PageClassification{Unclassified}
%
% Optional fields for the SF298.
%
\RPTpreparedFor{}
\ContractNumber{}
\GrantNumber{}
\ProgramElementNumber{}
\TaskNumber{}
\WorkUnitNumber{}
\POReportNumber{}
\Acronyms{}
\SMReportNumber{}
\SubjectTerms{}
\ResponsiblePerson{}
\RPTelephone{}
\SignatureOne{}
\SignatureTwo{}
\SupplementaryNotes{The views expressed in this document are those of
  the author and do not reflect the official policy or position of the
  Department of Defense or the U.S. Government. %
  IRB Protocol Number: N/A. % if you need to note an IRB Protocol or N/A
}

% Optional. Prevents footnotes from being reset at each chapter
% Comment this out to have them reset with each chapter.
\makeatletter
\@removefromreset{footnote}{chapter}
\makeatother

% Optional. Adds pdf metadata and links.
% This should be right before the \begin{document}, to be the
% last package / macros defined. (Hyper-ref is fragile,
% needs to be last, and has known conflicts with other packages.)
% Comment out if you have build problems building with hyperref
\NPShyperref

%
% Your thesis begins here
%
\begin{document}

\NPScover                  % Cover page
\NPSsftne                  % SF298 form
%\NPSsignature             % Tech Report page (iii): signature page
\NPSthesistitle            % Thesis page (iii): title page
\NPSabstractpage           % Abstract Page
\NPSfrontmatter            % NPS front matter follows

% This changes the chaptermark and includes the various tables
% It must be here.
\renewcommand{\chaptermark}[1]{\markboth{\MakeUppercase{\chaptername}\ \thechapter.\ #1}{}}

%
% If you don't need one of these, comment it out.
%
\NPStableOfContents
\NPSlistOfFigures
\NPSlistOfTables
\NPSlistOfAcronymsFromFile{acronyms}

%
% Put Executive summary here.
% New paragraphs start after an empty line.
%
\NPSexecsummary{
\begin{bibunit}[nps_thesis] % START: bibunit, use nps_thesis.bst for the style

This is an example of how to create an executive summary with its own references 
section using the \texttt{bibunit} package.
The build process needs to change to accomodate this. 
The \texttt{bibunit} package builds separate unit files
\texttt{bu1.aux}, \texttt{bu2.aux}, etc. 
These needs to be run through Bib\TeX{} separately.
In this example, the executive summary if the first (and only) bibunit, 
so we need to do the commands:
\begin{itemize}
	\item[] \texttt{pdflatex report}
	\item[] \texttt{bibtex report}
	\item[] \texttt{bibtex bu1}
	\item[] \texttt{pdflatex report}
	\item[] \texttt{pdflatex report}
\end{itemize}
The \texttt{Makefile} demonstrates how to script this.

\section*{Executive Summary Section}
The references~\cite{IEEEexample:incollectionmanyauthors},
\cite{IEEEexample:articledualmonths}
are in both the summary,
and in the final references; they are numbered separately.
Some references~\cite{IEEEexample:shellCTANpage,IEEEexample:bibtexuser,
IEEEexample:article_typical} only appear in this section,
and are not part of the final bibliography.

Note, you cannot use numbered sections in the executive summary,
since the summary has no number itself.

This sentence demonstrates that acronyms, like \ac{US}, work in the
executive summary; the \ac{US} is the short version. The counter will be re-set
in the main body, where its first use will be long again.

\subsection*{An Exec Summary Subsection}
\lipsum[2-3] % example text; remove me

%
% this makes references appear at the section-level instead of chapter-level
%
\begingroup
 \let\stdthebibliography\thebibliography
 \renewcommand{\thebibliography}{%
 \let\chapter\subsubsection
 \titlespacing*{\subsubsection}{0pt}{5ex plus 2ex}{-1ex plus .2ex}
 \stdthebibliography}
 \raggedright     % don't automatically full justify bibliographic references
 \singlespacing   % reduce extra line breaks between entries
 \small
 \putbib[thesis]  % use thesis.bib and place the bibliography here
\endgroup
%
% This is the end of special macros that tweak the appearance of the references
%
\end{bibunit}    % END: bibunit
}

%
% Put acknowledgements here.  
% New paragraphs start after an empty line.
%
\NPSacknowledgements{%
\lipsum[1-3] % example text; remove me
}

% Start layout for the NPS body
\NPSbody


% CHAPTERS
% You have two options on how to structure your thesis:
% a) A single file. All chapters, sections, etc. go in this file.
%    This can make navigating your thesis a little more difficult.
% b) Use multiple files.  One chapter per file is recommended.
%    This breaks your thesis up into logical units to edit.
%
\chapter{[Chapter Title]}\label{ch:common}

This is the beginning of Chapter~\ref{ch:common}. Always have text between every head
and subhead. The most commonly used references by our community are
   TESTBalls~\cite{appleton2014additive},
   books~\cite{IEEEexample:book_typical},
   journal articles~\cite{IEEEexample:article_typical},
   conference proceedings~\cite{IEEEexample:conf_typical},
   online resources~\cite{IEEEhowto:IEEEtranpage},
   theses~\cite{IEEEexample:masters},
   private communication~\cite{IEEEexample:private}, and
   class notes~\cite{NPSexample:notes}.
Here are some citations\footnote{These citations were taken
from the IEEEtran example bib file.} that demonstrate nearly every bib 
style\footnote{\lipsum[10]} that exists~\cite{IEEEhowto:IEEEtranpage,
IEEEexample:shellCTANpage,IEEEexample:bibtexuser,IEEEexample:bibtexdesign,
IEEEexample:tamethebeast,IEEEexample:bibtexguide,
IEEEexample:article_typical,IEEEexample:articleetal,IEEEexample:conf_typical,
IEEEexample:book_typical,IEEEexample:articlelargepages,
IEEEexample:articledualmonths,IEEEexample:TBParticle,
IEEEexample:bookwitheditor,IEEEexample:book,IEEEexample:bookwithseriesvolume,
IEEEexample:inbook,IEEEexample:inbookpagesnote,IEEEexample:incollection,
IEEEexample:incollectionwithseries,IEEEexample:incollection_chpp,
IEEEexample:incollectionmanyauthors,
IEEEexample:motmanualhowpub,IEEEexample:confwithadddays,
IEEEexample:confwithvolume,IEEEexample:confwithpaper,
IEEEexample:confwithpapertype,IEEEexample:presentedatconf,
IEEEexample:masters,IEEEexample:masterstype,IEEEexample:phdurl,
IEEEexample:techrep,IEEEexample:techreptype,IEEEexample:techreptypeii,
IEEEexample:techrepstdsub,IEEEexample:unpublished,IEEEexample:electronhowinfo,
IEEEexample:electronhowinfo2,IEEEexample:electronorgadd,IEEEexample:uspat,
IEEEexample:jppat,IEEEexample:frenchpatreq,
IEEEexample:standard,IEEEexample:standardproposed,IEEEexample:draftasmisc,
IEEEexample:miscforum,IEEEexample:whitepaper,IEEEexample:datasheet,
IEEEexample:private,IEEEexample:miscrfc,IEEEexample:softmanual,
IEEEexample:softonline,IEEEexample:miscgermanreg,IEEEexample:bluebookstandard
}.

This shows the acronym macro being used for the \ac{US}, which can
be used again in its short form \ac{US}.

\section{Problem Statement}\label{sec:Problem Statement}
A simple equation to check numbering.
\begin{equation}
a^2 + b^2 = c^2
\end{equation}

stuff 
 
\lipsum[1-4] % remove me

\begin{figure}
\framebox[\textwidth]{\parbox{\textwidth}{\lipsum[1]}} % example figure (text in a box)
\caption[{[Alternate short name]}]{[Figure name]}
\end{figure}

\section{Another Section}
A simple equation to check numbering.
\begin{equation}
L' = {L}{\sqrt{1-\frac{v^2}{c^2}}}
\end{equation}
\lipsum[2-3] % remove me

\subsection{A subsection}
\lipsum[5-6] % remove me

\section{Yet Another Section}
\lipsum[1] % remove me
\begin{table}
\begin{center}
\begin{tabular}{ | c | c | c | }
\hline
  1 & 2 & 3 \\ \hline\hline
  4 & 5 & 6 \\
  7 & 8 & 9 \\
\hline
\end{tabular}\vspace{-1em}% remove some of the whitespace that pads the end
\end{center}
\caption{[Table name]}
\end{table}
\lipsum[2]

\subsection{A subsection with a long name that continues onto two lines and should be single-spaced within the title}\label{sec:another}
\lipsum[3]

\subsubsection{A subsubsection}\label{sec:minorstuff}
\lipsum[4]


\section{Things to Remember When Writing}\label{sec:remember}
\begin{enumerate}
\item Punctuation (periods and commas) go inside quotation marks. 
\item Use the \LaTeX{} \verb+\begin{figure}+ and \verb+\begin{table}+ environment to
  create floating figures and tables. Use the \verb+\caption+ command
  to create your captions. Label your captions with the
  \verb+\label{foo}+ command inside the caption itself. Reference
  these figures and tables with the \verb+\ref{foo}+ reference command.
\item The macros \verb+\etal+, \verb+\ie+, \verb+\eg+ and \verb+\etc+ force proper
  U.S. convention for use of these, \ie a comma follows.
  It is redundant and incorrect to use \etc at the end of a list of
  examples, \eg, apples, pears, \etc.
\item Do not split text around a figure or table. 
\item Master's degree has an apostrophe and Postgraduate is one word. 
\item If you use ``however,'' make sure there's a comma before and after,
unless you 
start a sentence with it. However, it's best not to start a sentence
with ``however.'' And while we're on the subject, you should try to avoid starting a sentence with ``and'' or ``because.'' 
\item When typing a date, do not use ``st'' or ``th.'' Instead, just
  note the date: July 4, 1776, is Independence Day. Commas go 
after month/date, year: both Jefferson and Adams died on July 4, 1826.
No comma between month/yr: \textit{Alice's Adventures in Wonderland} was published in July 1865.
\item In general, spell out numbers one through nine, and use numerals for 10 and greater.
\item Use automatic numbering and lettering.
\item Capitalize C in Chapter, F in Figure and T in Table when referring
to chapters, figures or tables in the text. Better yet, use the \verb+\chapref+,
\verb+\figref+ and \verb+\tabref+ commands in the NPS report template.
\item When there is more than one reference, put them both into the \verb+\cite+ command: \verb+\cite{john1,john2}+. It will render like this \cite{IEEEhowto:IEEEtranpage,IEEEexample:shellCTANpage}.
\item Avoid writing in the first person!
\item Make sure there are no widows at the
  top of the page---if \LaTeX{} gives you a hard time, you may need to
  add or remove text so that everything works out properly.
\item Footnote numbers go outside the punctuation. 
\item When typing equations in text, use ``where'' or ``if.'' Use
  Math Mode. 
\item When inserting symbols, use Math Mode.
\end{enumerate}
%
% (include other chapters here...)
%


% APPENDICES
% You have two recommended options for your appendix:
% a) A single appendix (with a single TOC entry)
% b) Multiple appendices. Look under the examples directory for a demo of
%   multiple appendices.
%
\NPSappendixTOC{[My Appendix Title]}
\lipsum*[65]
\begin{eqnarray}
 e^x &\approx& 1+x+x^2/2! + \\
   && {}+x^3/3! + x^4/4! + \\
   && + x^5/5!
\end{eqnarray}

\section{Section example}
\lipsum[47]

\subsection{Subsection example}
\lipsum[56]

\section{Another section}
\lipsum[55-56]

\begin{figure}
\framebox[\textwidth]{\parbox{\textwidth}{\lipsum[65]}}
\caption{Some styled text in a caption, \emph{emph} and \textit{italics} in a caption.}
\end{figure}

\begin{figure}
\framebox[\textwidth]{\parbox{\textwidth}{\lipsum[65]}}
\caption{Some styled math in a caption, $\mathsf{Func}(x, z) = x^2 + \overline{z} + \pi$.}
\end{figure}

\begin{figure}
\centering
\subfigure[First sub-figure]{
   \framebox[0.47\textwidth]{\parbox{0.45\textwidth}{\lipsum[65]}}
}
\hfill
\subfigure[Second sub-figure]{
   \framebox[0.47\textwidth]{\parbox{0.45\textwidth}{\lipsum[65]}}
}
\caption{Caption using subfigure package.}
\end{figure}


% REFERENCES
% List all your BibTeX reference source files (ending in *.bib extension)
%
\NPSbibliography{thesis}


%
% This is the official end of the thesis.
%
\NPSend

% DISTRIBUTION LIST
% The list is automatically properly numbered
% and already populated with the mandatory recipients.
%
\NPSdistribution{Initial Distribution List}
\begin{distributionlist}
\item Defense Technical Information Center\\Ft. Belvoir, Virginia
\item Dudley Knox Library\\Naval Postgraduate School\\Monterey, California
%
%---- Other entries are no longer needed, because of Special Abstract Form
% Marine Corps students are required to show:
%\item Marine Corps Representative\\Naval Postgraduate School\\Monterey, California
%\item Directory, Training and Education, MCCDC, Code C46\\Quantico, Virginia
%\item Marine Corps Tactical System Support Activity (Attn: Operations Officer)\\Camp Pendleton, California
%
% Officer students in the Operations Research Program are also required to show:
%\item Director, Studies and Analysis Division, MCCDC, Code C45\\ Quantico, Virginia
%
% Officer students in the Space Ops/Space Engineering Program or in the Information Warfare/Information Systems and Operations are also required to show:
%\item Head, Information Operations and Space Integration Branch,\\ PLI/PP\&O/HQMC, Washington, DC
\end{distributionlist}


\end{document}